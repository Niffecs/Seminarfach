\chapter{Geschichte der Kryptografie}
Die Geschichte der Kryptographie kann man in drei Epochen unterteilen.
Die erste Epoche umfasst alle Verschlüsselungsmethoden die manuell, also schriftlich durchgeführt wurden.
Diese begann mit den ersten Verschlüsselungsmethoden, welche dreitausend vor Christus entwickelt worden und endete mit dem Ausgang des Ersten Weltkrieges. 
Die zweite Epoche umfasst die Zeit von 1920 bis etwa 1970. 
In ihr veränderte sich die Art und Weise der Verschlüsselung deutlich, da nun nicht mehr manuell, sondern maschinell verschlüsselt wurde. 
So kamen nun spezielle Verschlüsselungs-Maschinen, die Rotor-Chiffriermaschinen, zum Einsatz. 
In der dritten Epoche wurden die Maschinen von Computern abgelöst, welche die Nachrichten nun nicht mehr maschinell, sondern nur noch digital verschlüsselten. 
Diese Entwicklung begann circa 1970, mit der Verbreitung der ersten Computer und hält bis heute an. 
Dies war nur ein kurzer Ausblick auf meinen Teilbereich der Seminarfacharbeit, der Geschichte der Kryptographie. 
Genaueres zu den jeweiligen Epochen werden Sie in meiner Ausarbeitung zu diesem Thema lesen können. 

Die erste Epoche der Kryptographie beginnt im Altertum. Denn bereits im dritten Jahrtausend vor Christus wurden in Ägypten Nachrichten verschlüsselt. Dies basiert auf dem Prinzip des Rebus und der Akrophonie, welches eine Verschlüsselung in Form eines Bilderrätzels beschreibt. In dieser Zeit wurden die altägyptischen mythologisch-religiösen Texte von Priestern verschlüsselt, da die öffentliche Aussprache von verschiedenen Gottheiten verboten war. Weiter Varianten der Verschlüsselung im Altertum, war die Cäsar-Chiffre, welches eine Verschiebung von Buchstaben im Alphabet um einen festgelegten Wert beschreibt. Aber auch die Griechen und Mesopotamier nutzen bereits verschieden Methoden zur Verschlüsselung. Durch das Mittelalter kam die Forschung im Bereich der Kryptographie in Europa zum Erliegen, wodurch kaum neue Methoden zur Verschlüsselung aus dem europäischen Raum überliefert worden sind. Die einzigen Überlieferungen aus dem europäischen Raum waren Forschungen und Abhandlungen eines englischen Mönches und Universalgelehrte namens Roger Bacon. Doch in der arabischen Welt gab es in der Zeit zwischen 500 und 1400 nach Christus, einen starken Anstieg in der Kryptographie-Forschung, welche jedoch erst gegen Ende des 20. Jahrhunderts entdeckt wurde und in der Forschung berücksichtigt wurde. Einer der Hauptvertreter war der islamische Theologe und Philosoph, Abū Ya\'qūb ibn Ishāq al-Kindī, welcher als Erster statistische Methoden zur Kryptoanalyse beschrieb. Nachdem die Forschung in der Kryptographie mehrere Jahrhunderte kaum vorangetrieben wurde, bekam sie mit dem Beginn der Renaissance einen erheblichen Aufschwung. Da in dieser Zeit, die kaum veränderten Verfahren, weiterentwickelt und verbessert wurden. Diese Entwicklungen und Forschungen im Bereich der Kryptographie, wurden zu Beginn der Neuzeit hauptsächlich in Italien gemacht, weshalb sie in dieser Zeit zur führenden Kryptographie-Nation wurden. 

Eine der bedeutendste Entwicklung war die Chiffrierscheibe, welche 1466 vom Italiener Leon Battista Alberti entwickelt wurde. Diese Scheibe besteht aus zwei runden Platten, die sich auf einer gemeinsamen Achse befinden und so verbunden sind, dass die kleinere Scheibe auf der größeren Schreibe liegt und sich drehen kann. Auf der kleineren Scheibe befindet sich das Alphabet mit dem Klartext und auf der äußeren Scheibe befindet sich das Alphabet mit dem Geheimtext. Damit ist eine monoalphabetische Verschlüsselung erreicht wurden oder, wenn man die Stellung der Scheiben zueinander während der Verschlüsselung verändert, sogar eine polyalphabetische Verschlüsselung, was bei komplexeren Varianten, dieser Scheiben, zu finden ist.\\

Das 19. Jahrhundert war eine der ersten Hochzeiten in der Kryptographie, da sich in dieser Zeit nicht nur Experten in diesem Gebiet versuchten, sondern auch einfache Menschen aus der Gesellschaft. Dadurch sind viele unterschiedliche Chiffren entstanden, wie zum Beispiel die Beale-Chiffre oder die Dorabella-Chiffre. In diesen wurde oft der Weg zu einem großen Goldschatz oder anderen Reichtümern vermutet, doch nach jahrelangen Forschungen von Experten, sind viele dieser Chiffren für ungültig erklärt worden, da der Inhalt ohne Sinn ist und somit zu keinem Ziel führt. In anderen Fällen, ist auch heute noch keine Entschlüsselung gefunden wurden, womit sie auch ungültig wurden oder als nicht lösbar eingestuft wurden. 
Ein bekanntes Beispiel für solch eine unlösbare Chiffre, ist das Voynich-Manuskript. Dies ist ein 224 Seiten dickes Buch, welches in einer unbekannten Sprache verfasst wurde und in einer unnatürlichen Sprache. Dadurch wurde vermutet, dass es sich um einen verschlüsselten Text handeln muss. Doch durch neuste Forschungen konnte festgestellt werden, dass es sich um eine sinnlose Buchstabenfolge handeln muss. Durch die Entwicklung und Verbreitung des Telegrafen, musste es zu neuen Überlegungen bzw. zu neuen Arten der Verschlüsselung in der Kryptographie kommen, da die Telegrafen auf einfachste Weise angezapft und abgehört werden konnten. So kam der Kryptologe Auguste Kerckhoffs von Nieuwenhof auf einen noch heute gültigen Grundsatz in der Kryptographie. Das Kerckhoffs Prinzip besagt, dass die Sicherheit eines kryptographischen Verfahrens allein auf der Geheimhaltung des Schlüssels basiert, aber das Verfahren an sich kann öffentlich zugänglich sein. Dies soll den Zweck haben, dass Forscher und Experten dieses Verfahren untersuchen und verbessern können, damit so neue Varianten zur Verschlüsselung entstehen können.\\


Der erste Weltkrieg war ein Wendepunkt in der Nutzung der Kryptographie, da in dieser Zeit, erstmals alle beteiligten Großmächte erkannten, wie wichtig die Geheimhaltung von Informationen gegenüber dem Feind war. Dadurch begannen alle Kriegsparteien ihre Nachrichten bestmöglich zu verschlüsseln. Dies war zu Anfang des Krieges noch recht einfach, da einige beteiligte Staaten noch keine bzw. kaum Experten zur Entschlüsselung feindlicher Funkspruche besaßen.\\



Diese Zahl stieg im Laufe des Krieges deutlich an. Trotz des Anstieges und der Erforschung neuer Verschlüsselungsverfahren, konnten fast alle verwendeten Verfahren mit recht wenig Aufwand geknackt werden. Eines der bekanntesten entschlüsselten Nachrichten aus dem ersten Weltkrieg ist die Zimmermann-Depesche, in der das Deutsche Reich, Mexiko Gebiete in den USA versprach, wenn sie sich den Deutschen anschließen. Dadurch sah sich die USA gezwungen in den Krieg einzutreten. Im ersten Weltkrieg waren die Verfahren zur Ver- und Entschlüsselung von Nachrichten noch schriftlich und eine der Bekanntesten Verfahren, war das „ADFGX“ der Deutschen oder auch die Geheimschrift der Funker genannt. Dies war ein manuelles Verschlüsselungsverfahren, welches über die drahtlose Telegrafie Nachrichten übermittelt. Es basiert auf einer zweistufigen Verschlüsselung, die durch Substitution, die Ersetzung von Zeichen durch andere Zeichen und gefolgt von einer Transposition, die Vertauschung der Anordnung der Zeichen, passiert.\\

Nachdem der erste Weltkrieg beendet war, erkannte man das manuelle Verschlüsselungsverfahren zu langsam und zu ineffizient geworden sind, wodurch die Entwicklung von Verschlüsselungsmaschinen begann. Dadurch wurde die zweite Epoche in der Geschichte der Kryptographie eingeläutet. Die Epoche der Verschlüsselung mit Maschinen. Eine der ersten Verschlüsselungsmaschinen war das One-Time-Pad, welche 1918 veröffentlicht wurde. Als Erfinder dieser Maschine, gilt der Ingenieur Gilbert Vernam, da er die Idee zu solch einer Maschine hatte. Doch der Amerikaner Joseph O. Mauborgne, war der, der diese Idee umsetzte. In diesem Verfahren wird der Text zeichenweise gemeinsam mit einer zufälligen Zeichenfolge verschlüsselt, die nur einmal verwendet wird. Das One-Time-Pad war die erste populäre Verschlüsselungsmaschine, da man das Verfahren des Pads per Maschine, aber auch per Hand nutzen konnte. Die Maschine wurde auch noch im Zweiten Weltkrieg, als auch im Kalten Krieg am Roten Telefon, die ständige telefonische Verbindung zwischen der Sowjetunion und der USA, während des Kalten Krieges, verwendet. In dieser Zeit wurde die Rotor-Chiffriermaschine erfunden, wodurch sie zu der Hauptverschlüsselungsmaschine des Zweiten Weltkrieges wurde. Nach dem Ausbruch des Zweiten Weltkrieges begann, wie schon im Ersten Weltkrieg, eine Art Wettrüsten in Sachen Verschlüsselungsmaschinen und Verschlüsselungstechniken. So besaß jede Großmacht ihre eigene Maschine. Dadurch wurde auf deutscher Seite die „ENIGMA“, auf amerikanischer die „M-209“ und die „SIGBA“, auf britischer die „TypeX“ und auf sowjetischer die „Fialka-Maschine“ eingesetzt. Die wohl bekannteste, aber auch wichtigste Entschlüsselung in der Geschichte der Kryptographie wurde wohl gegen Ende des Zweiten Weltkrieges in England gemacht. Denn dort entwickelte der britischen Ingenieur Harold Keen und seinem Team aus zwölf Mitarbeitern der „British Tabulating Machine Company“ den ersten Prototypen für die Turing-Bombe, welche für die Entzifferung von deutschen Enigma-Funksprüchen zuständig war.\\


Nachdem die Effizienz der Maschine, durch den Einsatz eines Diagonalbrettes, welches von Gordon Welchman erfunden wurde, immer weiter verbessert werden konnte, wurde die Produktion deutlich erhöht. Somit gab es bis Kriegsende 210 solcher Maschinen, die auch Bombes genannt wurden. Die erste betriebsfähige Turing-Welchmann-Bombe, welche 15 Minuten für das vollständige Absuchen des Schlüsselraumes, auch Exhaustion genannt, einer Walzenlage benötigte. Die Dauer konnte bei späteren Modellen sogar auf 6 Minuten reduziert werden. Als das Standard-Enigma-Modell gegen das neuere und bessere Enigma-M4-Modell, welche nun 4 Walzen hatte, ausgetauscht wurde, musste die Turing- Welchmann- Bombe verbessert werden, da diese nur die Standard-Enigma knacken konnte. Erst durch diese Turing-Bombe konnten die Funksprüche der deutschen Wehrmacht zuverlässig entschlüsselt werden und waren somit maßgeblich am Sieg der alliierten Mächte im Zweiten Weltkrieg beteiligt. Nach dem Ende des Zweiten Weltkrieges neigte sich die Ära der maschinellen Verschlüsselung dem Ende hin. Dies wurde durch das Aufkommen des Computers Anfang der 70er Jahre deutlich, da durch den wachsenden Datenverkehr die Nachfrage nach Datenverschlüsselung und Datensicherheit immer größer wurde. Desweitern wandelte sich die Kryptographie von einer reinen Geheimwissenschaft zu einer Forschungsdisziplin, die auch in der Öffentlichkeit betrieben wurde. Deshalb wurde 1976 der DES- Algorithmus (Data Encryption Standard) von IBM und der NSA entwickelt. Dieser Algorithmus stellte einen einheitlichen und sicheren Standard für behördenübergreifende Verschlüsselungen dar. Der DES und sicherere Varianten finden auch heute noch Verwendung wie zum Beispiel bei Bankdienstleistungen. Ein weiterer großer Fortschritt wurde mit dem Veröffentlichen des Public-Key- Verfahrens gemacht. Den Grundstein für dieses Verfahren legten Whitfield Diffie und Martin Hellman, auch Diffie-Hellman-Schlüsselaustausch genannt, in ihrem Artikel „New Directions in Chryptography“. Dieser Artikel stellte eine neue Methode der Schlüsselverteilung dar und gab somit den Anstoß zur Entwicklung des Public-Key- Verfahrens. Dieser Diffie-Hellman-Schlüsselaustausch war das erste Public-Key- Kryptoverfahren und war somit auch das erste Asymmetrische Kryptosystem. Die Innovation hinter diesem Verfahren war, das zwei Personen über eine öffentliche Leitung, mit einem selbstgewählten Schlüssel auf ein gemeinsames Ziel zugreifen können. Diese Funktionsweise findet auch heute noch in der Schlüsselverteilung der Kommunikations- und Sicherheitsprotokolle des Internets, besonders im Onlinehandel, seine Anwendung. Dadurch hat dieses Verfahren auch heute noch einen hohen Stellenwert. Das bekannteste der Public-Key- Verfahren ist aber das RSA-Kryptosystem, welches 1977 von Ronald L. Rivest, Adi Shamir, Leonard Adleman veröffentlicht wurde. Vor der Entwicklung des Asymmetrischen Kryptosystems waren Schlüssel noch symmetrisch, das heißt, mit ihnen konnte man die Nachricht sowohl Entschlüsseln als auch Verschlüsseln. 


Dies machte das Verschlüsseln bzw. Entschlüsseln unsicher, da man die Schlüssel sicher zum Kommunikationspartner bringen musste. Da durch den immer stärker verbreiteten Computer der Datenverkehr immer mehr zunahm, womit das Symmetrische Kryptosystem immer unübersichtlicher wurde. Da die Anzahl an Kommunikationspartnern immer weiter anstieg. Desweitern wurde für jeden weiteren Kommunikationspartner ein neuer Schlüssel benötigt, da so nur verhindert werden konnte, dass ein anderer die Nachricht entschlüsseln kann. Dieses Verfahren nennt man Private-Key-Verfahren. Obwohl die neuere Verschlüsselungsmethode, die asymmetrische RSA-Verschlüsselung, zwar sicherer war als die symmetrische DES-Verschlüsselung, war sie trotzdem langsamer. Anfang der 1990er wurde der Computer und das Internet immer mehr von der breiten Masse genutzt. Dadurch kam der Wunsch auf, dass man auch private Nachrichten verschlüsseln kann. Dieses Privileg, dass sie Verschlüsselungen nutzen durften, hatten bis dahin nur Regierungen und Großunternehmen mit leistungsstarken Rechnern. Daraufhin entwickelte der amerikanische Physiker Phil Zimmermann eine RSA- Verschlüsselung, die auch von Privatleuten genutzt werden konnte. Diese Verschlüsselung nannte sich Pretty Good Privacy, kurz PGP. Das Verfahren wurde 1991 veröffentlicht und mit ihr war es nun erstmals möglich, dass Nachrichten unter Privatleuten, mit Hilfe des Public-Key-Verfahren, verschlüsselt versendet werden konnten. Nach zwei Jahrzehnten in Benutzung suchte man 1997 nach einem Nachfolger für den DES- Standard. Dieser sollte nun nicht mehr von der NSA entwickelt werden, sondern fort an von unabhängigen Kryptologen. Deren Vorschläge wurden auf zwei Konferenzen im Jahr 1998 und 1999 ausgewertet. In der letzten Konferenz im Jahr 2000 wurde der neue „Advanced Encryption Standard“, kurz AES, vom National Institute of Standards and Technology vorgestellt. Er hieß Rijndael und zeichnete sich durch seine deutlich höhere Geschwindigkeit, im Gegensatz zu den anderen Vorschlägen, aus. Bei diesem neuen Standard handelt es sich um ein symmetrisches Verschlüsselungsverfahren. Auf diesen Grundlagen, bauen auch heute noch die Verschlüsselungen auf, nur das sie immer weiter verbessert worden und vor allem auch sicherer geworden sind. Doch genaueres zu den Funktionsweisen der aktuellen Verschlüsselungsmethoden können sie im Kapitel „Asymmetrische Verschlüsslung“ lesen. 


\section{Fazit}

Die Kryptographie ist meiner Meinung nach eine der wichtigsten Wissenschaften, da diese mehrere Bereiche, wie Mathematik, Maschinenbau, aber auch Informatik vereint. Denn erst durch das mathematische Grundverständnis von Substitution und Transposition, konnten Verschlüsselungsmethoden entwickelt werden, auf denen die heutigen Methoden aufbauen. Diese Methoden konnten mit der Entwicklung des Computers auch den Themenbereich der Informatik abdecken, wodurch die Verbindung zwischen Mathematik und Informatik entstehen konnte. Im Laufe der Zeit wurde aus einer reinen Wissenschaft, eine weitverbreitete Freizeitaktivität, wovon auch die breite Öffentlichkeit profitierte. Denn dadurch, dass die Kryptographie nun nicht mehr ausschließlich für wissenschaftliche Zwecke genutzt wurde, sondern auch für die Allgemeinheit, wodurch nun auch Verschlüsselungsmethoden für die Bevölkerung entwickelt wurden. So entstand zum Beispiel das „PGP“, die „Pretty Good Privacy“, womit nun auch erstmals Nachrichten zwischen Privatleuten verschlüsselt werden. Doch man sollte nicht übersehen, dass es die Wissenschaft der Kryptologie bereits mehrere tausend Jahre gibt, doch die wirklich wichtigen Entwicklungen und Erfindungen, aber erst in den letzten 125 Jahren getätigt wurden. In der Zeit davor, war die Kryptographie eher eine Randerscheinung, wenn nicht sogar fast vergessen, wie es in Europa zu Zeiten des Mittelalters der Fall war. Erst mit dem Beginn der Renaissance, welche gegen Ende des 14. Jahrhunderts anfing, erlebte die Kryptographie eine Art Wiedergeburt, die in Europa von Italien ausging. Ab diesem Zeitpunkt aber, entwickelte sich diese Wissenschaft immer weiter. So konnte man die Verfahren immer komplexer und sicherer gestalten, da durch die Weiterentwicklung der Technik immer mehr Leistung für die Verfahren zur Verfügung standen. Somit war es Anfang der 1920er erstmals möglich, eine Verschlüsselung von einer Maschine durchführen zulassen. Dadurch wird die Wissenschaft des Maschinenbaus mit der Wissenschaft der Kryptographie vereint.


\section{Vergleich der Epochen in der Kryptographie}
In dem folgenden Text, möchte ich die einzelnen Epochen der Kryptographie miteinander vergleichen. Was möchte ich vergleichen, ich möchte das jeweilige Ende der Epoche mit dem Ende der darauffolgenden Epoche vergleichen. So werde ich zum einen das wichtigste, die Geschwindigkeit zum Verschlüsseln der Nachricht, dann in welcher Art und Weise die Nachricht verschlüsselt wird, des Weiteren wie sich die Maschinen äußerlich verändert haben und die Sicherheit der Verschlüsselungsmethode, vergleichen.\\



\subsection{Vergleich Ende 1. Epoche mit Ende 2. Epoche}
Das Ende der 1. Epoche ist im Zeitraum des 1. Weltkrieges zu finden, also Anfang des 20. Jahrhunderts. In dieser Zeit wurden die Grundlagen entwickelt, welche die Basis für die Verschlüsselungsmaschinen der darauffolgenden 2. Epoche waren. Doch erst durch den Ersten Weltkrieg wurde die Forschung in der Kryptographie vorangetrieben, da alle Beteiligten des Krieges ihre Nachrichten bestmöglich vor der Gegenseite verschlüsseln wollten, damit keine wichtigen Informationen zum Feind durchsickern konnten. Das Ende der 2. Epoche, ist im Zeitraum des 2. Weltkrieges und des beginnenden Kalten Krieges anzusiedeln. Gefördert durch den 2. Weltkrieg, durchlebte die Kryptographie einen stetigen Fortschritt in Sachen Geschwindigkeit zum Ver- und Entschlüsseln von Nachrichten, wodurch vor allem die Komplexität der Verschlüsselung anstieg und somit auch eine höhere Sicherheit gewährleistet werden konnte. Dieser Anstieg konnte, durch die Entwicklung von speziellen Rotor-Chiffriermaschinen, ermöglicht werden. Diese Entwicklung basierte auf den Grundlagen, die in der ersten Epoche geschaffen worden waren. 

\subsection{Art und Weise der Verschlüsselung Ende der 1. und 2. Epoche}

Anfang des 20. Jahrhunderts wurde die Verschlüsselung noch hauptsächlich manuell, also per Hand durchgeführt. So wurden Handschlüssel, aber auch Codebücher zur Ver- und Entschlüsselung von Nachrichten verwendet. Bei den Codebüchern handelt es sich um ein Verzeichnis in dem Buchstaben, Ziffern, Wörter und sogar ganze Sätze mit ihren entsprechenden Zeichenkombinationen aufgelistet waren. Somit war die Textlänge im Telegramm kürzer und auch effizienter. Unter einem Handschlüsselverfahren versteht man eine Verschlüsselung durch eine Substitution (das ersetzen von Zeichen durch andere) und darauffolgend, eine Transposition (das Vertauschung der Anordnung der Zeichen). Dieses Verfahren wurde mit der deutschen ADFGX (Geheimschrift der Funker) eingesetzt. Zur Übermittlung von Daten wurde die drahtlose Telegrafie angewandt. Nachdem im 1. Weltkrieg die Verschlüsselungen noch hauptsächlich per Hand getätigt worden waren, wurden nun spezielle Maschinen zum Verschlüsseln entwickelt und verwendet. Die bekannteste aus dieser Zeit, ist die deutsche Verschlüsselungsmaschine „Enigma“. 
Diese war eine Rotor-Schlüsselmaschine (Rotor-Chiffriermaschine), wie die meisten Verschlüsselungsmaschinen des Zweiten Weltkrieges. Die Funktionsweise einer solchen Maschine muss man sich wie folgt vorstellen. Die Rotoren sind als drehbare Walzen angeordnet, und ihre Stellung zueinander ändert sich während des Schlüsselvorgangs. An ihren Außenflächen besitzen sie mehrere Kontakte (oft genau 26 für die 26 Großbuchstaben des lateinischen Alphabets), die im Inneren durch isolierte Drähte miteinander verbunden sind. Durch die Drehung der Rotoren wird für jeden Buchstaben des Textes eine unterschiedliche Ersetzung erzeugt. Die kryptographische Sicherheit der Verschlüsselung hing wesentlich von der Anzahl der verwendeten Rotoren ab, da damit die Menge der möglichen Ersetzungen (auch Schlüsselraum genannt) multiplikativ anstieg. 

\subsection{Aussehen der Verschlüsselungsmethoden Ende der 1. und 2. Epoche}

In dieser Zeit kann man noch nicht von einem wirklichen Design sprechen, da die Hauptverschlüsselungsmethoden schriftlich erledigt wurden und somit nie ein einheitliches Äußeres zustande kam. Nach dem der Erste Weltkrieg beendet wurde, änderte sich auch das Aussehen der Verschlüsselungsmaschinen wesentlich. So wurden Anfang der 1920er die ersten Maschinen speziell zur Verschlüsselung entwickelt. Die damaligen Maschinen kann man vom Äußeren her mit einer Schreibmaschine vergleichen, da die Maschinen auch ein Eingabefeld bzw. eine Art Tastatur besaßen. Der Unterschied zu einer Schreibmaschine liegt im Inneren einer solchen Verschlüsselungsmaschine, da sich dort die Rotoren zur Ver und Entschlüsselung befinden.\\

\subsection{Geschwindigkeiten der Verschlüsselungsmethoden Ende der 1. und 2. Epoche}
Dadurch, dass in dieser Epoche die Ver- und Entschlüsselung hauptsächlich per Hand durchgeführt wurde, kann man keinen bestimmten Wert für die Geschwindigkeit festlegen. Die Geschwindigkeit richtete sich immer nach der Leistung des Ausführeden der Verschlüsselung, aber auch die Länge der Verschlüsslung spielt hierbei eine Rolle. Grundsätzlich kann man sagen, dass die Verschlüsselungsmaschinen im Vergleich deutlich schneller und auch effizienter waren, als alle bisherigen schriftlichen Methoden. Desweitern konnte man durch den Einsatz von Maschinen, im Gegensatz zu den schriftlichen Methoden, nun auch viel komplexere Nachrichten erstellen und verarbeiten. Desweitern war es nun möglich, deutlich mehr Nachrichten am Stück zu verschlüsseln bzw. zu entschlüsseln, da nun die Maschine den Rechenanteil übernimmt und nicht mehr, wie in der ersten Epoche, ein Mensch die Rechnung übernehmen muss.\\


\subsection{Sicherheit der Verschlüsselungsmethoden Ende der 1. und 2. Epoche}
Von großer Sicherheit konnte man in der Zeit des Ersten Weltkrieges noch nicht sprechen, da die bisherige Ver- und Entschlüsselung von Nachrichten noch per Hand, also manuell geschah. Dadurch musste die Verschlüsselung noch so einfach gestaltet sein, damit der Mensch die Nachricht, ohne sehr großen Zeitaufwand fehlerfrei entschlüsseln konnte. Desweitern war der Übermittler von Nachrichten hauptsächlich der Telegraf, welcher ohne viel Aufwand abgehört werden konnte. Bei größeren Texten nutzte man im Ersten Weltkrieg zur Entschlüsselung von Nachrichten, Codebücher, welche als Wörterbücher zur Verschlüsselung dienten. Doch auch diese Nachrichten waren nicht unknackbar, da im Laufe des Krieges einige Codebücher dem Feind in die Hände fielen und somit auch diese Nachrichten entschlüsselt werden konnten. Dadurch war es in dieser Zeit sehr schwierig seine Nachrichten von Absender zum Empfänger komplett geheim zu halten. Nach dem in der 2. Epoche, Maschinen speziell zur Ver- und Entschlüsselung von Nachrichten entwickelt wurden, konnte man nun erstmals von einer gewissen Sicherheit sprechen. Da man nun nicht mehr auf die Fähigkeiten des Menschen, sondern auf die der Maschine setzte. Dadurch konnten die Nachrichten deutlich aufwendiger und komplexer verschlüsselt werden. Durch den Einsatz von Maschinen wurde das Ver- und Entschlüsselung auch schneller, wodurch eine Nachricht nun schneller geknackt bzw. chiffriert werden konnte.\\

\section{Vergleich Ende 2. Epoche mit Ende 3. Epoche}
Wie bereits genannt, wurde in der zweiten Epoche der Kryptographie, Maschinen zur Ver- und Entschlüsselung von Nachrichten verwendet. Mit der Entwicklung des Computers für den Massenmarkt, Anfang der 70er Jahre, wurden die rein maschinellen Verschlüsselungsmaschinen schnell verdrängt, da sie langsamer und dadurch ineffizienter waren, als die Computer. Somit änderte sich auch das Aussehen der Maschinen deutlich. So entwickelte es sich von einer rein maschinell arbeitenden und langsamen Maschine, zu einem digital arbeitenden Computer, welcher schneller und leistungsfähiger war. Da die Nachfrage für Computer schnell stieg, mussten auch hier neue Verschlüsselungsmethoden entwickelt werden. So wurde nun nicht mehr wie in der 2. Epoche üblich, Nachrichten mit einer Rotor-Chiffriermaschine verschlüsselt, sondern mit einem digitalen Algorithmus. Dadurch änderte sich auch die Art und Weise der Verschlüsselung. Der Erste Algorithmus, welcher die geforderte nötige Sicherheit bereitstellen konnte, war der DES (Data Encryption Standard). Eine der letzten entwickelten Methoden, welche auch heute noch verwendet wird, ist die RSA-Verschlüsselung. Diese Verschlüsselungsmethode ist deutlich sicherer, als die der 2. Epoche, weil sie das asymmetrische Verfahren verwendete, welches in Komplexität und Leistungsfähigkeit deutlich umfangreicher war, als dass Substitutionsverfahren der Rotor-Chiffriermaschine.

\subsection{Fazit}
So wie sich der Mensch im Laufe der Zeit immer weiterentwickelt hat, so hat sich auch die Kryptographie weiterentwickelt. Dadurch, dass das Wissen des Menschen immer größer wurde, stieg auch die Angst vor Diebstahl des geistigen Eigentums, wodurch der Grundgedanke der Kryptographie geschaffen wurde. Das erste Mal, dass dieser Gedanke aufkam, war dreitausend vor Christus, bei den Ägyptern. Dies trieb die Menschen an, immer neue und bessere Methoden zur Verschlüsselung zu entwickeln. Genau diese stetige Weiterentwicklung und Verbesserung der Verschlüsselungsmethoden wird in meinem Vergleich deutlich. Dadurch, dass ich die Enden der jeweiligen Epochen mit meinen selbst gewählten Vergleichspunkten, welche Sicherheit, Geschwindigkeit, Art und Weise und das Aussehen umfassen, gegenüberstellt habe, konnte ich nachvollziehen, wie sich die Methoden zur Verschlüsselung, im Laufe der Zeit verändert haben. So wurde mir klar, dass es erst mit dem Ende des Ersten Weltkrieges, die Anzahl von neuen Verschlüsselungsmethoden ständig anstieg bzw. die Forschung an neuen Varianten begonnen hat. Dies war auch die Zeit, in der die Kryptographie eine Art Revolution durchlief, da es einen Wechsel von manueller Verschlüsselung zu maschineller Verschlüsselung gab. Dieser Prozess wiederholte sich mit der Entwicklung des Computers, womit man von der maschinellen Verschlüsselung zur digitalen Verschlüsselung überging. Von nun an sind auch die Zeiträume zwischen dem Wechsel einer Epoche deutlich kleiner geworden, so war die erste Epoche noch mehrere tausend Jahre lang, die zweite Epoche nur 50 Jahre lang und die aktuelle besteht sogar erst seit 1970.