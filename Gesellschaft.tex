\chapter{Kryptographie in Recht und Gesellschaft}
Nachdem wir uns mit der Geschichte, Verschlüsselungsverfahren und explizit mit der RSA Verschlüsselung und deren Sicherheit beschäftigt haben, kommen wir nun zu den rechtlichen Rahmenbedingungen. Sie sind besonders wichtig in Bezug auf das Kapitel „Datenschutz und Datensicherheit“ zu nennen. \\

In der heutigen modernen Zeit findet man Sie überall, die Verschlüsselung. Sei es bei Sozialen Netzwerken oder bei Smartphones. Doch wie sieht es in den Bereichen Recht und Gesellschaft aus?

\section{Rechtliche Aspekte}
\subsection{Situation in den USA}

In den USA ist es bis jetzt nicht gesetzlich geregelt, ob die Benutzung leistungsfähiger kryptographischer Verfahren strafbar ist oder nicht. Bisher unterliegt solche Hard- und Software besonderen Exportbeschränkungen und werden wie Munition behandelt. Deshalb sind für den Vertrieb in andere Länder, nur Verfahren zugelassen die maximal eine Schlüssellänge von 40 Bit besitzen. Doch diese verfahren können sehr leicht und schnell geknackt werden.

\subsection{Situation in Dänemark}
In Dänemark wird das Recht auf Verschlüsselung nicht durch ein Gesetz eingeschränkt, denn dort hat jeder das Recht darauf seine persönlichen Daten zu verschlüsseln. Wenn Firmen die Kryptographie als Grundsatz ihrer Dienste sehen, müssen Sie die Verschlüsselung, durch einen Gerichtsbeschluss entschlüsseln. Da der dänische Technologierat festgestellt hat, dass ein gänzliches Verbot von Verschlüsselungssystemen gegen den Artikel 8 und Artikel 10 der Europäischen Menschenrechtskonvention verstößt. Somit sollten Behörden keinen Zugriff auf die Schlüssel der Programme erhalten.

\subsection{Situation in Deutschland}
\subsubsection{Die Aktuelle Rechtslage}
Authentikations- und Konzelationssystemen werden im deutschen Gesetzgeber getrennt bestrafft, obwohl beide Systeme kryptographische Verfahren sind. Während das Authentikationssystem eine detaillierte gesetzliche Regelung besitzt, so ist das Konzelationssystem kaum bis gar nicht gesetzlich geregelt. Die Unterscheidung von Signatur- und Verschlüsselungsverfahren entsteht dadurch, dass sie aus juristischer Sicht verschiedenen Zielen dienen und das die Erwartungen der einzelnen Benutzern an die Systeme aus unterschiedlichen Problemen zusammensetzen.


\subsubsection{Die Gesetzgebungskompetenz}
Die Zuständigkeit der Gesetzgebung ergibt sich aus der Kompetenz des Bundes zu konkurrierenden Gesetzen der Wirtschaft (Art.74 Abs. 1 Nr. 11 GG). Damit in gesamt Deutschland die gleichen Sicherheitsbedingungen und Sicherheitsanforderungen herrschen können, ist die Einheit von Recht und Wirtschaft durch eine bundesgesetzliche Regelung gewahrt (Art. 72 Abs. 2 GG).

\subsubsection{Das Signaturgesetz}
Das Gesetz zur Behandlung der Rahmenbedingungen für Informations- und Kommunikationsdienste (Informations- und Kommunikationsdienste-Gesetz, IuKDG) vom 22. Juli 1997 enthält im Art. 3 das Gesetz zur digitalen Signatur (Signaturgesetz, SigG). Dieses Gesetz ist nach dem Art. 11 IuKDG am 1. August 1997 in Kraft getreten.

\subsubsection{Gesetzeszweck}
Die Absicht des SigG ist es, Rahmenbedingungen für digitale Unterzeichnungen zu befördern, unter denen diese als gewiss gelten und Fälschungen digitaler Signaturen oder Verfälschungen von unterzeichneten Daten zuverlässig diagnostiziert werden können (§ 1 Abs. 1 SigG). Veränderungen an Formvorschriften oder dem Beweisrecht wurden nicht vorgenommen, hauptsächlich wurde die digitale Signatur nicht der gesetzmäßigen Schriftform (§ 126 BGB) gleichgestellt. Es soll gegenwärtig durch tatsächliche Sicherheit Vertrauen in die gesetzliche digitale Signatur hervorgerufen werden, sodass sie vom Rechtsverkehr angenommen wird und Gerichtshöfe ihr im Rahmen der freien Beweiswürdigung die nötige Beweiskraft zusprechen können. Anschließend wird dann möglicherweise in bestimmten Fällen, in denen im Augenblick Schriftform verlangt wird, auch die digitale Signatur zugelassen werden; außerdem könnte es dann auch Abwandelungen im Beweisrecht geben. Das SigG stellt die Verwendung anderer Verfahren in § 1 Abs. 2 ausdrücklich frei; es besteht also kein Zwang, nur Authentikationssysteme nach dem SigG zu gebrauchen.

\subsubsection{Die Zertifizierungsstellen}
Das SigG klärt im Wesentlichen die Rahmenbedingungen für die Arbeit mit Zertifizierungsinstanzen, die Zertifizierungsstellen genannt werden. Nach § 2 Abs. 2 SigG sind dies Menschen, die die Überreichung von allgemein zugänglichen Prüfschlüsseln an natürlichen Personen beglaubigen und dafür eine Genehmigung nach § 4 SigG besitzen. Die Prüfschlüssel der Zertifizierungsstellen werden nach § 4 Abs. 5 SigG ihrerseits von der berechtigten Behörde zertifiziert, sodass sich eine zweistufige Zertifizierungshierarchie mit der Verwaltung an der Spitze ergibt. Mit der Einigung, dass alle Zertifizierungsstellen von der zuständigen Behörde zu zertifizieren sind, wird ein wesentliches Sicherheitsrisiko geschaffen. Sollte der Schlüssel des Amtes kompromittiert werden, so ist die ganze Zertifizierungshierarchie und damit der gesamte elektronische Rechtsverkehr schutzlos. Daneben wird aber auch verhindert, dass Zertifizierungsstellen „Dependancen“ veranlassen können und sich somit eine mehrstufige Zertifizierungshierarchie entwickeln kann. Es ist fraglich, ob dies zweckentsprechend ist. \\

Mithilfe des SigG und die aufgrund von § 16 SigG durch die Bundesregierung vorgeschriebene Verordnung zur digitalen Signatur (Signaturverordnung, SigV) vom 22. Oktober 1997, die Detailfragen klärt und laut § 19 SigV am 1. November 1997 in Kraft getreten ist, werden den Zertifizierungsstellen eine Anzahl von Pflichtaufgaben übergeben. Die Zertifizierungsstelle muss die Identität einer Person, die ein Zertifikat anfordert, durch Abnahme des Personalausweises diagnostizieren (§§ 5 Abs. 1 SigG, 3 Abs. 1 SigV). Sie muss für den Bittsteller ein Schlüsselpaar anfertigen und es ihm zuordnen oder aber sich davon überzeugen, dass der Antragsteller zur Empfängnis des Schlüsselpaars angemessene technische Komponenten eingesetzt hat (§ 5 SigV). Hat sie das Schlüsselpaar erzeugt, muss sie es dem Antragsteller persönlich überreichen (§ 6 SigV). Sie stellt ein Prüfschlüssel-Zertifikat aus (§ 5 Abs. 1 Satz 2 SigG), das weitere Daten wie Angaben über Vertretungsmacht oder Berufszulassungen aufweisen kann (§ 5 Abs. 2 SigG), und belehrt den Antragsteller über Sicherheitsfragen (§§ 6 SigG, 4 SigV). Daneben muss die Zertifizierungsstelle einen immer erreichbaren Sperrdienst (§§ 8 SigG, 9 SigV), einen 10 Jahre langen öffentlich zugänglichen Telekommunikationsverbindungen erreichbaren Verzeichnisdienst für Zertifikate und Sperrungen (§§ 5 Abs. 1 Satz 2 SigG, 8 SigV) sowie einen Dienst zum „Zeitstempeln“ digital unterschriebener Daten (§ 9 SigG) betreiben.

\subsubsection{Die Lizenzierung}
Nach § 4 SigG bedarf der Betrieb einer Zertifizierungsstelle einer Lizenz. Es besteht ein Rechtsanspruch auf Erteilung einer Lizenz durch die Behörde, wenn der Antragsteller die für den Betrieb einer Zertifizierungsstelle nötige Zuverlässigkeit und die dort tätigen Personen die nötige Fachkunde besitzen und außerdem in einem Sicherheitskonzept und bei einer Vorortprüfung die Erfüllung der technischen und organisatorischen Sicherheitsanforderungen des SigG nachgewiesen werden. Die Lizenz kann mit erforderlichen Nebenbestimmungen versehen werden (§ 4 Abs. 4 SigG). Aufgrund eines Zirkelschlusses zwischen § 2 Abs. 2 SigG, der in die Legaldefinition einer „Zertifizierungsstelle“ als Voraussetzung den Besitz einer Lizenz nach § 4 SigG aufnimmt, und § 4 Abs. 1 SigG, der für den Betrieb einer Zertifizierungsstelle eine Lizenzierungspflicht aufstellt, ist unklar, ob eine Lizenz für den Betrieb jeder Zertifizierungsinstanz erforderlich ist, oder ob sich nur diejenigen lizenzieren lassen müssen, die Zertifikate nach dem SigG ausstellen und damit an der gesetzlichen Sicherheitsfiktion des § 1 Abs. 1 SigG teilhaben wollen. Der Wortlaut von § 4 Abs. 1 SigG, der sich eines im selben Gesetz legaldefinierten Begriffs bedient, und die grundsätzliche Freistellung alternativer Verfahren für die digitale Signatur in § 1 Abs. 2 SigG sprechen für die liberalere Auslegung. Die Bundesregierung wollte jedoch offenbar jegliche Ausstellung von Zertifikaten für Signaturprüfschlüssel einer Lizenzierungspflicht unterwerfen. Dieser Wunsch hat im maßgeblichen Gesetzestext allerdings keinen Ausdruck gefunden. Nimmt man die Legaldefinition in § 2 Abs. 2 SigG ernst, so ergibt sich, dass die Lizenzierung eine freiwillige Unterwerfung unter die Bestimmungen des SigG darstellt und keine Pflicht dazu besteht, wenn eine Zertifizierungsinstanz die Vorteile des SigG (wie etwa die Fiktion in § 1 Abs. 1 SigG) nicht in Anspruch nehmen möchte. Diese Auslegung allein ist auch in der Lage, den durch § 1 Abs. 2 SigG angestrebten Wettbewerb des gesetzlichen Modells mit anderen Sicherungsinfrastrukturen zu gewährleisten, da ein Authentikationssystem ohne Zertifizierungsstruktur im großen Maßstab nicht einsetzbar ist. Die dem experimentellen Charakter des SigG entsprechende Erprobung von Technologien, von denen noch nicht bekannt ist, ob sie den Sicherheitsanforderungen des SigG entsprechen, wird nur so ermöglicht. Auch wird damit die absurde Konsequenz vermieden, dass praktisch jeder Benutzer des im Internet gebräuchlichen Verschlüsselungsprogramms PGP als „Zertifizierungsstelle“ eine Lizenz nach § 4 SigG benötigte. Und schließlich verlangt das rechtsstaatliche Erfordernis der Klarheit und Bestimmtheit eine Regelung, die für den Einzelnen erkennen lässt, was von ihm verlangt wird und inwiefern seine Grundrechte eingeschränkt werden. \\
Eine Gesetzesauslegung, die einen Eingriff in die Berufsfreiheit (Art. 12 Abs. 1 GG) oder die allgemeine Handlungsfreiheit (Art. 2 Abs. 1 GG) darauf stützen wollte, dass sie eine durch das selbe Gesetz aufgestellte Legaldefinition für unbeachtlich hält, müsste sich vorwerfen lassen, dieses Erfordernis aufzugeben. Demnach ist der Auslegung der Vorzug zu geben, nach der keine Lizenzierungspflicht besteht.


\subsubsection{Die Datenschutzaspekte}
Personenbezogene Daten, die die Zertifizierungsstelle nach § 12 Abs. 1 SigG erhoben hat, unterliegen einer Zweckbindung. Der Antragsteller ist nicht einmal gezwungen, sein Zertifikat veröffentlichen zu lassen; es genügt nach § 5 Abs. 1 Satz 2 SigG, wenn eine Online-Überprüfung bei der Zertifizierungsstelle möglich ist. Nach § 5 Abs. 3 SigG sind auch Zertifikate, die anstelle eines Namens mit einem Pseudonym versehen sind, zulässig, wobei dieser Umstand im Zertifikat kenntlich zu machen ist; auf Wunsch des Kunden muß ein solches Zertifikat sogar ausgestellt werden. Jedoch ermächtigt § 12 Abs. 2 SigG Sicherheits- und Geheimdienstbehörden, die wahre Identität eines Zertifikat-Inhabers bei der Zertifizierungsstelle zu ermitteln, soweit das zur Erfüllung ihrer Aufgaben notwendig ist. Diese Regelung ist in zweierlei Hinsicht zu kritisieren. Zum einen ist keine Möglichkeit für Privatpersonen vorgesehen, um eine Aufdeckung der pseudonymen Identität zu erreichen. Dies wird die tatsächlichen Einsatzmöglichkeiten von pseudonymen digitalen Signaturen erheblich behindern, da sich ein Vertragspartner normalerweise auf einen Abschluß mit einem unter Pseudonym agierenden Partner nur einlassen wird, wenn er bei Leistungsverweigerung die wahre Identität in Erfahrung bringen und auf dem Rechtsweg gegen seinen Vertragspartner vorgehen kann. Zum anderen ist die Regelung aber auch zu weitgehend. Die Aufdeckung der pseudonymen Identität durch eine Behörde stellt einen Eingriff in das informationelle Selbstbestimmungsrecht des Betroffenen aus Art. 1 Abs. 1, 2 Abs. 1 GG dar. Die Eingriffsvoraussetzungen in § 12 Abs. 2 SigG sind allerdings zu unbestimmt und es gibt auch keine besonderen verfahrensmäßigen Sicherungen, was beides den von BVerfG vorgegebenen Bedingungen widerspricht. Insbesondere ist nicht einmal eine Unterrichtung des Betroffenen nach Ende der Ermittlungen vorgesehen, so daß dieser möglicherweise noch lange auf die Geheimhaltung seiner pseudonymen Identität vertraut, während der Staat weiterhin in der Lage ist, seine Aktionen zu verfolgen. Dadurch geht der Eingriff in zeitlicher Hinsicht weiter als erforderlich und wird damit unverhältnismäßig. In seiner jetzigen Fassung ist § 12 Abs. 2 SigG aus diesen Gründen verfassungswidrig.

\subsubsection{Sicherheit und Überprüfungen}
Die Paragraphen §§ 5 Abs. 4, 5 Abs. 5 SigG, 10 f., 14 ff. SigV enthalten Ausführungen, die technische, organisatorische und personelle Sicherheit versichern sollen. Besondere Geltung hat das Gesetz der Aufbewahrung des persönlichen Signaturschlüssels bei der Zertifizierungsstelle in § 5 Abs. 4 Satz 3 SigG, denn wer über diesen Schlüssel verfügt, kann hypothetisch willkürliche Dokumente mit der digitalen Signatur des Schlüsselinhabers versehen; diese Eventualität würde dem Vertrauen in die digitale Signatur schweren Beeinträchtigen zufügen. Die Ausführungsbestimmungen, die eine Zertifizierungsstelle zur Einhaltung der Sicherheitsanforderungen durchführt, sind in einem Sicherheitskonzept niederzuschreiben (§§ 4 Abs. 3 Satz 3 SigG, 12 SigV). Dieses Manuskript wird bei der Lizenzierung und danach jeweilig im Abstand von zwei Jahren kontrolliert (§ 15 SigV). Außerdem hat die berechtigte Behörde zur Überwachung der Zertifizierungsstelle Durchsuchungs-, Einsichts- und Auskunftsrechte. Sie kann Maßnahmen gegen die Zertifizierungsstelle verhängen und die Lizenz entweder zurücknehmen oder aberkennen (§ 13 SigG). Daneben hat die Datenschutzbehörde gemäß § 12 Abs. 3 den Anspruch zu verdachtsunabhängigen Untersuchungen. § 13 Abs. 2 SigG sieht zwar vor, dass die Kontrollbehörde bei einer Überprüfung Einsicht in die schriftlichen Materialien der Zertifizierungsinstanz nehmen kann, doch fehlt eine Bevollmächtigung zum Vollzug auf technische Anlagen und Datensammlungen. Eine allumfassende Überprüfung scheint ohne diese Eventualität annäherungsweise ausgeschlossen.


\subsubsection{Die Haftungsfragen}
Das SigG enthält keine persönliche Haftungsregelung und überlässt die Beantwortung von Haftungsfragen den allgemeingültigen Regeln. Demzufolge verpflichtet sich die Zertifizierungsstelle gegenüber dem Schlüsselinhaber gemäß § 278 BGB ohne Exkulpationsmöglichkeit für ein Verschulden ihrer Angestellter und anderweitigen Erfüllungsgehilfen aus positiver Forderungsverletzung des Zertifizierungsvertrags, wobei dem Schlüsselinhaber die allgemeine Beweislastverteilung entsprechenden Verantwortungsbereichen (Rechtsgedanke des § 282 BGB) zugutekommt: er ist verpflichtet nur die Pflichtverletzung und deren Ursächlichkeit für den Schaden zu demonstrieren. Gegenüber Dritten haftet die Zertifizierungsstelle für wesentliche Vermögensschäden jedoch nur bei beabsichtigten Beeinflussungen durch einen ihrer Mitarbeiter (§ 831 BGB iVm § 826 BGB bzw. §§ 823 Abs. 2 BGB, 263 StGB), und dies mit Exkulpationsmöglichkeit. Damit entsteht sich eine Haftungslücke zumindest für Fälle, in denen eine leichtsinnige Verhaltungsweise von Mitarbeitern oder Ausführungen Dritter einen Vermögensschaden bei einem anderen als dem Schlüsselinhaber herbeigeführt haben.\\

Es wurde vorgeschlagen, zur Behebung dieser Lücke eine Gefährdungshaftung mit Haftungshöchstgrenze, Deckungsvorsorge und Ursachenvermutung einzuführen. Dies würde jedoch für Zertifizierungsstellen eine schärfere Einstandspflicht als für Notare nahelegen, obgleich die von letzteren geschriebenen Beglaubigungen und Beurkundungen im Wiederspruch zu digitalen Signaturen und Zertifikaten authentischen Glauben besitzen. Auch wegen einiger ausgedehnterer dogmatischer und Wertungswidersprüche sollte auf die Unterweisung einer derartigen Gefährdungshaftung dementsprechend verzichtet werden.


\subsubsection{Das Verbot starker Kryptographie}
Bereits weniger ausschlaggebend wäre ein Verbot annähernden kryptographischer Methoden, die vom Staat nicht mehr in zufriedenstellender Zeit entschlüsselt werden können, etwa durch die Beschränkung der zulässigen Schlüssellänge. Auch hier wäre wahrscheinlich ein Erlaubnisvorbehalt vorgesehen, sodass in besonders sensiblen Bereichen im Ausnahmefall doch einflussreiche Kryptographie zur Verwendung kommen könnte. Die Angemessenheit einer derartigen Ausführung ist annäherungsweise zu beurteilen wie die des Totalverbotes. Zwar ist bei diesem Modell im Fall einer Überwachung die umfassende Kommunikation gegenwärtig zu entschlüsseln, doch ist dies bei übereinstimmender wirtschaftlich finanziellen Ausstaffierung und Festlegung von minimalen Schlüsselhöchstlängen keine besondere Angelegenheit. Eine solche Behandlung wäre also gerade noch akzeptabel, um den anvisierten Zweck zu bewerkstelligen. Die Erforderlichkeit ist auch bei diesem Produkt mit der Problematik behaftet, dass Unbefugte – hier zugegebenermaßen nur finanziell leistungsfähige wie z. B. organisierte Gesetzesbrecher – Zugriff auf zu beschützende Daten erlangen können. Es ist also ebenfalls gesetzwidrig, wenn es ein anderes Mittel gibt, dass zu einer ausnahmslos milderen Beschuldigung führt.

\section{Gesellschaftliche Aspekte}
\subsection{Digitale Währung}
Mit digitalem Geld ist im Gegensatz zum elektronischen Zahlungsverkehr ein System gemeint, in dem digital versinnbildlichte Zahlungsmittel ohne Einschaltung Dritter von einer Person an eine andere exportiert werden können. Das Funktionieren eines derartigen Systems wäre eine weitere wichtige Grundsätzlichkeit für die Etablierung einer hochkarätigen Handelsinfrastruktur im Internet. Diese Zahlungsmöglichkeit sollte einige Voraussetzungen erfüllen: Das Geld kann über Computernetze exportiert werden, es kann nicht reproduziert und dann wiederverwendet werden, niemand kann zusammenfassend die Geschäfte eines Verbrauchers nachvollziehen, prinzipiell wird damit die Anonymität der Benutzer aufrechterhalten. Zur Begleichung der Schulden muss keine Verbindung zu einem Zentralrechner bestehen, das Geld kann zu anderen Benutzern übertragen werden und eine Bezahlung über einen bestimmten Betrag läßt sich in spärlichere Einheiten auseinandernehmen. Daneben ist für die Durchsetzung von digitalem Geld auch von Bedeutung, dass es kostengünstig, überall einsetzbar, einfach zu benutzen und auf einem robusten System basierend bewerkstelligt ist. Sogenannte elektronische Portemonnaies, die ihr Einsatzgebiet außerhalb des Netzwerkes im gewöhnlichen Leben finden sollen, müssen daneben auch über Vorteile gegenüber gebräuchlichem Bargeld verfügen, damit sie bei den Verbrauchern Anklang finden werden. Hier sind vor allem die vorangegangene erwähnte Teilbarkeit von Münzen, Verlusttoleranz, Quittierung und Stornierung von Zahlungen sowie Geldwechsel zwischen unterschiedlichen Währungen zu benennen. Es existieren währenddessen sehr komplizierte kryptographische Aufzeichnungen, die viele oder alle dieser Anforderungen verwirklichen und sich somit im Großen und Ganzen zur Umsetzung in die Praxis zur Verfügung stellen. Verschiedene Unternehmen ermöglichen digitales Geld auch zur Bezahlung im Internet an.


