\chapter{Zusammenfassung}
Die Kryptografie repräsentiert für uns ein sehr interessantes Feld, da noch nicht alles erforscht worden ist. Die Verschlüsselung von Daten hat einen sehr starken Bezug auf aktuelle Ereignisse, wobei der Abhörskandal um die NSA nur die Spitze des Eisberges darstellt. Ich sehe in der Kryptologie noch sehr viel Potential, vor allem durch das extrem starke Wachstum der Anzahl privater Daten. Verschlüsslung bietet keinen Schutz vor Diebstahl der Daten, aber vor dem unberechtigten Lesen Dritter. Schon das RSA Verfahren zeigt, wie einfach ein solches Verfahren sein kann. Welchen rasanten Fortschritt die Entwicklung nimmt, ist eigentlich gar nicht mit Worten zu beschreiben. Heute gilt RSA noch als sicher. Aber wie lange wird dies noch so sein? Schon 2013 sagte der damalige Bundesinnenminister Hans-Peter Friedrich „Verschlüsselungstechnik oder Virenschutz müsste mehr Aufmerksamkeit erhalten. Es gebe nun einmal die Möglichkeit zur Ausspähung, deshalb werde diese auch genutzt.“. Dies macht die große Nachfrage nach einer sicheren Verschlüsselung notwendig. Unserer Meinung nach, sollte ein jeder das Recht haben, seine eigenen und privaten Daten so verschlüsseln zu können, wie er es für richtig bzw. notwendig hält. Verschlüsselung in Kommunikationsanlagen sollte zum Standard werden. Dabei rede wir nicht nur von TLS und SSL. WhatsApp hat bereits eine Ende-zu-Ende Verschlüsslung eingeführt. Unser Erachten nach reicht das noch nicht. Mehrere Dienste sollten auf die erfolgreiche Entwicklung der Kryptologie eingehen und so die privaten Daten der Nutzer schützen. Aus heutiger Sicht sehe wir nicht nur RSA als sicher an. Auch andere symmetrische Systeme, wie DES und AES, sind sicher. Das ist unter anderem der Grund, warum sie noch heute Anwendung finden. Die Zukunft wird sich in Quantencomputern befinden. Auf diese konnten wir in der vorliegenden Arbeit nicht eingehen, da dies den zeitlichen und praktischen Rahmen gesprengt hätte. Sie handelt lediglich einen gesunden Rahmen der Kryptologie ab, angefangen bei der Geschichte bis hin zu modernen Verfahren und deren Sicherheit. Unserer Meinung nach bietet die Verschlüsselung noch viel Potential. Ein Zeichen dafür ist, dass in jeder geschichtlichen Epoche die Verschlüsselung eine bedeutende Rolle spielt. Unsere Umfrage zeigt auf, dass viele Menschen Verschlüsselungen kennen, sie aber nicht anwenden. Wenn es nach uns gehen würde, sollte jeder seine Daten ohne große Kenntnisse verschlüsseln können. Daraus resultiert auch der Eigenanteil. Objektiv betrachtet bietet unsere Seminarfacharbeit einen gesunden Ausblick in die Welt der Verschlüsselungen.